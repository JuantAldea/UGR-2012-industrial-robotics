% !TeX spellcheck = es_ES
%%%%%%%%%%%%%%%%%%%%%%%%%%%%%%%%%%%%%%%%%
% Short Sectioned Assignment
% LaTeX Template
% Version 1.0 (5/5/12)
%
% This template has been downloaded from:
% http://www.LaTeXTemplates.com
%
% Original author:
% Frits Wenneker (http://www.howtotex.com)
%
% License:
% CC BY-NC-SA 3.0 (http://creativecommons.org/licenses/by-nc-sa/3.0/)
%
%%%%%%%%%%%%%%%%%%%%%%%%%%%%%%%%%%%%%%%%%

%----------------------------------------------------------------------------------------
%	PACKAGES AND OTHER DOCUMENT CONFIGURATIONS
%----------------------------------------------------------------------------------------

\documentclass[paper=a4, fontsize=11pt]{scrreprt} % A4 paper and 11pt font size

\usepackage[spanish]{babel}
\usepackage[utf8]{inputenc}

\usepackage[T1]{fontenc} % Use 8-bit encoding that has 256 glyphs
\usepackage{fourier} % Use the Adobe Utopia font for the document - comment this line to return to the LaTeX default

\usepackage{amsmath,amsfonts,amsthm} % Math packages
\usepackage{sectsty} % Allows customizing section commands

\usepackage[pdftex]{graphicx}

\usepackage{listings}

\usepackage{calc}

\usepackage{appendix}

\allsectionsfont{\centering \normalfont\scshape} % Make all sections centered, the default font and small caps
\usepackage{float}
\usepackage{fancyhdr} % Custom headers and footers
\pagestyle{fancyplain} % Makes all pages in the document conform to the custom headers and footers
\fancyhead{} % No page header - if you want one, create it in the same way as the footers below
\fancyfoot[L]{} % Empty left footer
\fancyfoot[C]{} % Empty center footer
\fancyfoot[R]{\thepage} % Page numbering for right footer
\renewcommand{\headrulewidth}{0pt} % Remove header underlines
\renewcommand{\footrulewidth}{0pt} % Remove footer underlines
\setlength{\headheight}{13.6pt} % Customize the height of the header

\numberwithin{equation}{section} % Number equations within sections (i.e. 1.1, 1.2, 2.1, 2.2 instead of 1, 2, 3, 4)
\numberwithin{figure}{section} % Number figures within sections (i.e. 1.1, 1.2, 2.1, 2.2 instead of 1, 2, 3, 4)
\numberwithin{table}{section} % Number tables within sections (i.e. 1.1, 1.2, 2.1, 2.2 instead of 1, 2, 3, 4)

\setlength\parindent{0pt} % Removes all indentation from paragraphs - comment this line for an assignment with lots of text

\newlength{\imgwidth}

\makeatletter
\def\ScaleIfNeeded{%
    \ifdim\Gin@nat@width>\linewidth
    \linewidth
    \else
    \Gin@nat@width
    \fi
}
\makeatother

% Resize figures that are too wide for the page.
\let\oldincludegraphics\includegraphics

\renewcommand\includegraphics[2][]{%
    \oldincludegraphics[width=\ScaleIfNeeded]{#2}
}

%----------------------------------------------------------------------------------------
%	TITLE SECTION
%----------------------------------------------------------------------------------------

\newcommand{\horrule}[1]{\rule{\linewidth}{#1}} % Create horizontal rule command with 1 argument of height

\title{
    \normalfont \normalsize 
    \textsc{Robótica Industrial} \\ [25pt] % Your university, school and/or department name(s)
    \horrule{0.5pt} \\[0.4cm] % Thin top horizontal rule
    \huge Práctica 2: Cinemática: Método de Denavit-Hartenberg \\ % The assignment title
    \horrule{2pt} \\[0.5cm] % Thick bottom horizontal rule
}

\author{Juan Antonio Aldea Armenteros}

\date{\normalsize\today}

\begin{document}
    \maketitle
    \chapter{Ejercicio entregado en papel}
    Determinar los parámetros del método de D-H para el robot de la figura.
    \begin{figure}[H]
        \centering
        \oldincludegraphics[width=9cm]{./imagenes/dh-papel.png}
        \caption{Robot}
    \end{figure}
    \begin{table}[H]
        \centering
        \begin{tabular}{|c||c|c|c|c|}
            \hline
            Articulación & $\theta_i$ & $d_i$ & $a_i$ & $\alpha_i$ \\
            \hline
            \hline
            1 & $\theta_1$ & $0$ & $0$ & $-\frac{\pi}{2}$ \\
            \hline
            2 & $\theta_2$ & $l_1$ & $0$ & $\frac{\pi}{2}$ \\
            \hline
            3 & $\theta_3 - \frac{\pi}{2}$ & $0$ & $-l_2$ & $\frac{\pi}{2}$ \\
            \hline
            4 & $\theta_4$ & $l_3$ & $0$ & $-\frac{\pi}{2}$ \\
            \hline
            5 & $\theta_5$ & $0$ & $0$ & $\frac{\pi}{2}$ \\
            \hline
            6 & $\theta_6$ & $l_4$ & $0$ & $0$ \\
            \hline
        \end{tabular}
        \caption {Parámetros D-H calculados}
    \end{table}
    \newpage
    \chapter{Robot Bípedo}
        \begin{figure}[H]
            \centering
            \oldincludegraphics[width=4cm]{./imagenes/bipedo.png}
            \caption{Robot bípedo}
        \end{figure}
        Dado que el robot es simétrico solo es necesario calcular completamente los parámetros de una de las cadenas, pues se diferencian únicamente en una traslación y evidentemente en las variables que representan los grados de libertad, que son las correspondientes a cada cadena, de cualquier manera se incluyen las dos tablas.
    \begin{table}[H]
        \centering
        \begin{tabular}{|c||c|c|c|c|}
            \hline
            Art. & $\theta_i$ & $d_i$ & $a_i$ & $\alpha_i$ \\
            \hline
            \hline
            1 & $\theta_1 + \frac{\pi}{2}$ & $0$ & $0$ & $\frac{\pi}{2}$ \\
            \hline
            2 & $\theta_2$ & $0$ & $-l_{12}$ & $-\frac{\pi}{2}$ \\
            \hline
            3 & $\theta_3 - \frac{\pi}{2}$ & $-l_{23}$ & $0$ & $-\frac{\pi}{2}$ \\
            \hline
            11 & $\theta_11 + \frac{\pi}{2}$ & $-l_{34}$ & $0$ & $\frac{\pi}{2}$ \\
            \hline
            12 & $\theta_12 - \frac{\pi}{2}$ & $0$ & $l_{45}$ & $\frac{\pi}{2}$ \\
            \hline
            13 & $\theta_13$ & $0$ & $l_{56}$ & $0$ \\
            \hline
            14 & $\theta_14$ & $0$ & $l_{67}$ & $0$ \\
            \hline
            15 & $\theta_15$ & $0$ & $l_{78}$ & $-\frac{\pi}{2}$ \\
            \hline
            16 & $\theta_16$ & $0$ & $l_{89}$ & $0$ \\
            \hline
        \end{tabular}
        \begin{tabular}{|c||c|c|c|c|}
            \hline
            Art. & $\theta_i$ & $d_i$ & $a_i$ & $\alpha_i$ \\
            \hline
            \hline
            1 & $\theta_1 + \frac{\pi}{2}$ & $0$ & $0$ & $\frac{\pi}{2}$ \\
            \hline
            2 & $\theta_2$ & $0$ & $-l_{12}$ & $-\frac{\pi}{2}$ \\
            \hline
            3 & $\theta_3 - \frac{\pi}{2}$ & $-l_{23}$ & $0$ & $-\frac{\pi}{2}$ \\
            \hline
            4 & $\theta_4 + \frac{\pi}{2}$ & $-l_{34}$ & $0$ & $\frac{\pi}{2}$ \\
            \hline
            5 & $\theta_5 - \frac{\pi}{2}$ & $0$ & $l_{45}$ & $\frac{\pi}{2}$ \\
            \hline
            6 & $\theta_6$ & $0$ & $l_{56}$ & $0$ \\
            \hline
            7 & $\theta_7$ & $0$ & $l_{67}$ & $0$ \\
            \hline
            8 & $\theta_8$ & $0$ & $l_{78}$ & $-\frac{\pi}{2}$ \\
            \hline
            9 & $\theta_9$ & $0$ & $l_{89}$ & $0$ \\
            \hline
        \end{tabular}
        \caption {Parámetros D-H piernas (izquierda y derecha).}
    \end{table}
    \newpage
    \chapter{Robot Puma}
    \begin{figure}[H]
        \centering
        \oldincludegraphics[width=11cm]{./imagenes/puma.png}
        \caption{Robot PUMA con los ejes de coordenadas del método D-H}
    \end{figure}
    Teniendo en cuenta que el ángulo $\theta_i$ es la rotación sobre el eje $Z_{i-1}, 1 \leq i \leq 5$
    \begin{table}[H]
        \centering
        \begin{tabular}{|c||c|c|c|c|}
            \hline
            Art. & $\theta_i$ & $d_i$ & $a_i$ & $\alpha_i$ \\
            \hline
            \hline
            1 & $\theta_1$ & $13$ & $0$ & $-\frac{\pi}{2}$ \\
            \hline
            2 & $\theta_2$ & $0$ & $8$ & $0$ \\
            \hline
            3 & $\theta_3 - \frac{\pi}{2}$ & $-l$ & $0$ & $-\frac{\pi}{2}$ \\
            \hline
            4 & $\theta_4 + \pi$ & $8$ & $0$ & $-\frac{\pi}{2}$ \\
            \hline
            5 & $\theta_5 - \frac{\pi}{2}$ & $0$ & $0$ & $\frac{\pi}{2}$ \\
            \hline
            6 & $0$ & $t$ & $0$ & $0$ \\
            \hline
        \end{tabular}
        \caption {Parámetros D-H robot PUMA.}
    \end{table}
\end{document}
